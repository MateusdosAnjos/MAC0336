\documentclass[12pt]{article}

%These tell TeX which packages to use.
\usepackage{amsxtra}
\usepackage{amsthm}
\usepackage{amssymb}
\usepackage[utf8]{inputenc}

%New commands
\newcommand{\ceil[1]}{\lceil#1\rceil}
\newcommand{\modd}{\ \text{mod}\ }
\newcommand*{\QED}{\hfill\ensuremath{\blacksquare}}

%Pagination stuff.
\setlength{\topmargin}{-.3 in}
\setlength{\oddsidemargin}{0in}
\setlength{\evensidemargin}{0in}
\setlength{\textheight}{9.in}
\setlength{\textwidth}{6.5in}
\pagestyle{empty}

\begin{document}

	\begin{center}
		{\Large MAC0336/5723 Criptografia para Segurança de Dados\\
		Lista 3}\\
		\textbf{Mateus Agostinho dos Anjos}\\
		NUSP: 9298191
	\end{center}

	\vspace{0.4 cm}
	
	\subsection*{Exercício 1.}
		Dado os passos de 1 a 6 na página 182 temos:
		\begin{itemize}			
			\item[1 -]
				No passo 3 Beto escolhe um $0 < x < n$ e envia $a$ para
				Alice, tal que $x^2 \modd n = a$. No passo 4 Alice calcula as 
				quatro raízes quadradas de $a \modd n$ e envia uma delas 
				para Beto. \textbf{Ele ganha} caso \textbf{não receber} $x$ ou 
				$x - n$ (passo 5).\\
				A justificativa do porque Beto ganha e Alice aceita é explicada
				no passo 6. Se Beto receber outra raiz quadrada $y$ ou
				$n - y$ ele consegue fatorar $n$ com facilidade calculando
				$mdc(x+y, n) = p$ e envia a fatoração de $n$ para Alice, que
				aceita a vitória de Beto.
			\item[2 -]
				Alice ganha o jogo caso enviar para Beto $x$ ou $n - x$, pois
				com essas informações ele não consegue calcular a fatoração
				de $n$ e assim provar que ganhou. (Descrito no passo 5)
			\item[3 -]
				Sabemos que $n = pq$.\\		
				Assumindo que $y = 0$ fosse uma raiz válida que permitisse
				a Beto obter a fatoração de $n$, temos que $y \neq x$ e
				$y \neq n - x$.\\
				Desta forma Beto conseguiria calcular a fatoração de $n$ apenas com
				o $x$ escolhido por ele, uma vez que $mdc(x+y, n) = p$ ou $q$, 
				com $y = 0$ teríamos $mdc(x+0, n) = p$ ou $q$, ou
				seja $mdc(x, n) = p$ ou $q$, sendo que Beto conhece $x$ e $n$.
				Sendo assim, seria possível obter $p$ ou $q$ apenas com $mdc(x, n)$,
				favorecendo Beto que sempre ganharia o jogo, uma vez que a prova
				de sua vitória é obter a fatoração de $n$, pois mesmo que Alice 
				enviasse $x$ ou $n-x$, Beto ainda seria capaz de mostrar que ganhou.
			\item[4 -]
				$p = 3$, $q = 7$, $x = 4$, $a = ?$, $y = ?$, $mdc(x + y, n) = ?$\\
				$n = pq = 21$\\
				$x^2 \modd n = a$, portanto $4^2 \modd 21  = a$ então $a =  16$
				\newline
				\begin{center}
					Cálculo das raízes:\\
					$x_1 = a^{\frac{p+1}{4}} \modd p$	e 
					$x_2 = a^{\frac{q+1}{4}} \modd q$,
					sendo assim temos:\\
					$x_1 = 16^{\frac{3+1}{4}} \modd 3$, $x_1 = 1$\\
					$x_2 = 16^{\frac{7+1}{4}} \modd 7$, $x_2 = 4$\\
					Utilizando o Teorema Chinês do resto calcula-se $x_0$
					solução do sistema:
					$ \begin{cases} 
							x_0 = x_1 \modd p \\
							x_0 = x_2 \modd q 
						\end{cases}
					$\\
					Simplificando temos:\\
					$x_0 = (x_2pp^{-1} + x_1qq^{-1}) \modd pq$\\
					Calculamos $p^{-1}$ e $q^{-1}$ utilizando o algoritmo
					de Euclides estendido, chegando em:\\
					$p^{-1} = 5 $ e $q^{-1} = 1 $\\
					Portanto: $x_0 = (4*3*5 + 1*7*1) \modd 21$\\
					$x_0 = 4$\\
					Agora para o cálculo das outras 3 raízes temos:\\
					$x_0^{'} = (x_2pp^{-1} - x_1qq^{-1}) \modd pq$, $(pq - x_0)$,
					$(pq - x_0^{'})$\\
					$x_0^{'} = (4*3*5 - 1*7*1) \modd 21 = 11$\\
					$x_0^{''} = 21 - 4 = 17$\\
					$x_0^{'''} = 21 - 11 = 10$\\
					Pegando $y = 11$ temos $mdc(4+11, 21) = mdc(15, 21) = 3 = p$				
				\end{center} 	
				Terminado temos: 
				$p = 3$, $q = 7$, $x = 4$, $\mathbf{a = 16, y = 11, mdc(x + y, n) = 3}$,
				$n = 21$
		\end{itemize}
	\subsection*{Exercício 2.}
		\begin{itemize}
			\item[1 -]
				Sabemos que o testemunho $x = r^2 \modd n $\\
				Sabemos que $v = s^2 \modd n$\\
				$y$ é autêntico, portanto vale que:
				\begin{center}								
					$ \begin{cases} 
					y = r \modd n, & e = 0 \\ 
					y = rs \modd n, & e = 1
					\end{cases}
					$
				\end{center}
				Para $e = 0$:\\
				$xv^{e} \modd n = xv^{0} \modd n = x \modd n$\\
				\begin{center} 
					$y^2 = r^2 \modd n$\\
					$y^2 = x \modd n$\\
				\end{center}
				Vemos que:
				$ \begin{cases} 
					xv^{e} \modd n =  x \modd n \\ y^2 = x \modd n
					\end{cases}
				$\\
				Concluindo que, para $e = 0$ e $y$ autêntico, vale que $y^2 = xv^{e} 
				\modd n$\\
				\newline
				Para $e = 1$:\\
				$xv^{e} \modd n = xv \modd n$\\
				\begin{center} 
					$y^2 = (rs)^2 \modd n$\\
					$y^2 = r^2s^2 \modd n$\\
					$y^2 = (r^2 \modd n) \ (s^2 \modd n)$\\
					$y^2 = (x \modd n) \ (v \modd n)$\\
					$y^2 = xv \modd n$\\					
				\end{center}
				Concluindo que, para $e = 1$ e $y$ autêntico, vale que $y^2 = xv^{e}
				\modd n$
			\item[2 -]
				Para o protocolo de identificação Feige, Fiet e Shamir os parâmetros
				de segurança são:
					\begin{itemize}
						\item[•]
							O inteiro $\mathbf{s}$ relativamente primo a $n$, escolhido 
							por Alice, protegido pelo problema da fatoração de $n$, sendo 
							computacionalmente difícil calcular $s$ conhecendo-se apenas
							$v$ e $n$. O conhecimento de $\mathbf{s}$ facilitaria a
							personificação de Alice (no passo do envio de $y = rs$ para
							Beto) por algum mal intencionado.
						\item[•]
							O inteiro $\mathbf{r}$, protegido pela fatoração de $n$.
							Conhecendo-se $\mathbf{r}$ algum mal intencionado
							poderia enviar o testemunho $x$ para Beto, pois 
							$x = r^2 \modd n$ com $n$ conhecido e personificar
							Alice.
						\item[•]
							O desafio $e$ pode ser considerado um parâmetro de
							segurança, pois impede o ataque de um espião que mapeou
							todos os pares $x = r^2, y = rs$ a fim de responder
							$y = rs$ no passo 3, já que para $e = 1$ o passo 4
							seria $y^2 = xv = r^2s^2$. Com o desafio $e = 0$, 
							mapear todos os valores não auxilia o espião, já que a
							resposta exige $y = \sqrt{x} \modd n$ e fazer este
							cálculo sem a fatoração de $n$ é computacionalmente
							difícil. Portanto, pode-se dizer que o problema da fatoração
							de $n$ também protege a verificação quando é feito o
							desafio $e$.						
					\end{itemize}	
				Portanto conhecer $\mathbf{s}$ ou $\mathbf{r}$ facilita para um
				mal feitor personificar Alice, porém é necessário ter o conhecimento
				dos dois parâmetros para obter total sucesso na personificação.
			\item[3 -]
				O protocolo Feige, Fiet e Shamir é do tipo Zero Knowledge, pois
				permite a Beto verificar que é Alice verdadeira que manda as
				mensagens sem obter conhecimento sobre nenhuma informação
				privada dela, ou seja Beto não precisa saber qual a chave 
				$s$ utilizada por Alice para efetuar a verificação.
			\item[4 -]
				Para $t = 1$, $p = 3$, $q = 7$, $s = 17$, $r = 13$\\
				Temos:
				\begin{center}
					Cálculo de $n$:\\
					$n = pq$\\
					$n = 3*7$\\
					$\mathbf{n = 21}$\\
				\end{center}
				\begin{center}
					Cálculo de $v$:\\					
					$v = s^2 \modd n$\\
					$v = 17^2 \modd 21$\\
					$\mathbf{v = 16}$\\
				\end{center}
				\begin{center}
					Cálculo de $x$:\\					
					$x = r^2 \modd n$\\
					$x = 13^2 \modd 21$\\
					$\mathbf{x = 1}$\\
				\end{center}
				\newpage
				Para $e = 0$:
				\begin{center}
					Cálculo de $y$:\\					
					$y = rs^{e} \modd n$\\
					$y = 13*17^{0} \modd 21$\\					
					$\mathbf{y = 13}$\\
				\end{center}		
				\begin{center}
					Cálculo de $y^2$:\\					
					$y^2 \modd n$\\
					$13^2 \modd 21$\\
					$\mathbf{y^2 = 1}$\\
				\end{center}
				\begin{center}
					Cálculo de $xv^e \modd n$:\\					
					$1*16^0 \modd 21$\\
					$\mathbf{xv^e \modd n = 1}$\\
				\end{center}
				Verificando, portanto, que $y^2 = xv^e \modd n$ para $e = 0$\\
				Para $e = 1$:
				\begin{center}
					Cálculo de $y$:\\					
					$y = rs^{e} \modd n$\\
					$y = 13*17^{1} \modd 21$\\					
					$\mathbf{y = 11}$\\
				\end{center}		
				\begin{center}
					Cálculo de $y^2$:\\					
					$y^2 \modd n$\\
					$11^2 \modd 21$\\
					$\mathbf{y^2 = 16}$\\
				\end{center}
				\begin{center}
					Cálculo de $xv^e \modd n$:\\					
					$1*16^1 \modd 21$\\
					$\mathbf{xv^e \modd n = 16}$\\
				\end{center}
				Verificando, portanto, que $y^2 = xv^e \modd n$ para $e = 1$	
		\end{itemize}	
	\newpage				
	\subsection*{Exercício 3.}
		\begin{itemize}
			\item[1 -]
				Se $\mathit{s_A}$ for autêntico, então vale que:\\
				\begin{center}
					$\mathit{s_A} = (J_A)^{-s}$\\
					$J_A = (s_A)^{-v} \modd n$
				\end{center}
				Sabemos que, no protocolo de identificação vale que:\\
				\begin{center}
					$x = r^v \modd n$\\
					$y = r(s_A)^{e} \modd n$\\
					$z = J_A^{e} \ y^v \modd n$
				\end{center}
				Partindo de $z = J_A^{e} \ y^v \modd n$ temos:\\
				\begin{center}
					$z = J_A^{e} \ y^v \modd n$\\
					Substituindo $y$ por $ r(s_A)^{e} \modd n$:\\
					$z = J_A^{e} \  [r(s_A)^{e}]^v \modd n$\\
					Distribuindo o expoente $v$:\\
					$z = J_A^{e} \  r^v(s_A)^{e*v} \modd n$\\
					Unindo os termos com expoente $e$:\\
					$z = r^v [J_A \ (s_A)^{v}]^{e} \modd n$\\
					Substituindo $J_A$ por $(s_A)^{-v}$:\\
					$z = r^v [(s_A)^{-v} \ (s_A)^{v}]^{e} \modd n$\\
					Percebemos que o termo $[(s_A)^{-v} \ (s_A)^{v}]^{e}$ é igual
					a $[(s_A)^{-v + v}]^{e}$, ou seja $[(s_A)^{0}]^{e} = 1$\\
					$z = r^v \modd n$\\
					Como sabemos que $x = r^v \modd n$\\
					$z = x$\\
					$\QED$
				\end{center}
			\item[2 -]
				O caso $z = 0$ deve ser rejeitado, pois seria facilmente obtido
				por qualquer pessoa que escolhesse $r = 0$.\\
				Note que, se $r = 0$, então $x = r^v \modd n = 0$, no passo
				$y = r(s_A)^{e} \modd n$ se $r = 0$, independentemente de
				qual o segredo $(s_A)$, o valor de $y$ será 0, portanto, ao
				calcular $z = J_A^{e} \ y^v \modd n$, teríamos 
				$z = J_A^{e} \ 0^v \modd n$, logo $z = 0$ sem utilizar nenhuma
				informação que valide os parâmetros possuídos por Alice e chagando
				no resultado $z = x$ facilitando o trabalho de um invasor.
			\item[3 -]
				Pela escolha da entidade idônea $T$, $mdc[v, \Phi(n)] = 1$.\\
				A partir dos valores do item 3.7, temos $v = 11$ e
				$\Phi(n) = 72$, como 11 é primo e não divide 72, então o
				$mdc[v, \Phi(n)] = 1$ é verdadeiro e está verificado.
				\newpage
				Para $mdc[J_A, \Phi(n)] = 1$ temos $J_A = 29$ e $\Phi(n) = 72$,
				executando o algoritmo de Euclides chegamos em:\\
				$72/29 = 2*29 +  14$\\
				$29/14 = 2*14 + 1$\\
				$14/1 = 14 + 0$\\
				Portanto $mdc[29, 72] = 1$ é verdadeiro e está verificado .
			\item[4 -]
				As condições do mdc = 1 são exigidas para que haja inversa de $v$
				e de $J_A$, possibilitando o processo de verificação (protocolo
				de identificação).
			\item[5 -]
				Os parâmetros de segurança são:
				\begin{itemize}
					\item[]
						\textbf{Fatoração} de $n = pq$. Caso o intruso
						obtenha a \textbf{fatoração} de $n$, ou seja $p$ e $q$,
						o trabalho de recuperação
						da informação sobre as operações "mod n" e \\
						"mod $\Phi(n)$" são facilitadas.
					\item[]
						$\mathbf{s}$, protegido pelo problema da fatoração de $n$,
						assim o intruso não obtém $\Phi(n)$ além de ser 
						computacionalmente difícil recuperar $s$ a partir de
						$v^{-1} \modd \Phi(n)$. Caso o intruso conheça o $s$ ele
						pode calcular $s_A$ se grampear a informação $I_A$ e
						aplicar a função pública $f()$ em $I_A$, pois $J_A = f(I_A)$
						e $s_A = (J_A)^{-s} \modd n$.
					\item[]
						$\mathbf{s_A}$, protegido pela fatoração de $n$. Caso
						o intruso conheça $s_A$ ele pode escolher $r$ e fraudar 
						$y = r(s_A)^{e} \modd n$, se passando por uma falsa
						Alice.
				\end{itemize}
			\item[6 -]
				O protocolo é do tipo Zero Knowledge, pois Alice não revela
				nenhuma de suas informações particulares (secretas) para Beto.
				Ao final do processo Beto confirma que Alice conhece as chaves
				privadas, porém sem saber quais seus valores.
			\item[7 -]
				Para: $p = 7$, $q = 13$, $v = 11$, $J_A = 29$, $r = 13$, $e = 6$\\
				Calcular: $n$, $\Phi(n)$, $s$, $s_A$, $x$, $y$, $z$\\
				Verificar: $z = x$
				\begin{center}
					\begin{itemize}
						\item[]
							Cálculo de $n$:\\
							$n = pq$\\
							$n = 7*13$\\
							$\mathbf{n = 91}$
						\item[]
							Cálculo de $\Phi(n)$:\\
							$\Phi(n) = (p-1)(q-1)$\\
							$n = 6*12$\\
							$\mathbf{\Phi(n) = 72}$
						\item[]
							Cálculo de $s$:\\
							$s = v^{-1} \modd \Phi(n)$\\
							$s = 11^{-1} \modd \Phi(n)$\\
							$\mathbf{s  = 59}$
						\newpage	
						\item[]
							Cálculo de $s_A$:\\
							$s_A = (J_A)^{-s} \modd n$\\
							$s_A = 29^{-59} \modd 91$\\
							$s_A = 29^{32} \modd 91$\\							
							$\mathbf{s_A  = 22}$
						\item[]
							Cálculo de $x$:\\
							$x = r^{v} \modd n$\\
							$x = 13^{11} \modd 91$\\						
							$\mathbf{x  = 13}$
						\item[]
							Cálculo de $y$:\\
							$y = r(s_A)^{e} \modd n$\\
							$y = 13*(22)^{6} \modd 91$\\						
							$\mathbf{y  = 13}$
						\item[]
							Cálculo de $z$:\\
							$z = J_A^{e} \ y^{v} \modd n$\\
							$z = 29^{6} * 13^{11} \modd 91$\\						
							$\mathbf{z  = 13}$
					\end{itemize}	
				 \end{center}
			Conferimos que $x = z = 13$.	 
		\end{itemize}																		
	\subsection*{Exercício 4.}
		\begin{itemize}
			\item[1 -]
				Temos que $e = h(x||u)$ e $e' = h(x||z)$, portanto para mostrar
				que $e = e'$ basta mostrar que $u = z$.\\
				Temos que $u = b^r \modd p$, $v = b^{-s} \modd p$,
				$y = (s*e + r) \modd q$.
				\begin{center}
					Pela definição de $z$:\\
					$z = b^{y} \ v^e \modd p$\\
					Substituindo $y$ por $ (s*e + r) \modd q$:\\
					$z = b^{(s*e + r)} \ v^e \modd p$\\
					Distribuindo o expoente $(s*e + r)$:\\
					$z = b^{(s*e)} b^{r} \ v^e \modd p$\\
					Unindo os termos com expoente $e$:\\
					$z = (b^{s}v)^{e} b^{r} \modd p$\\
					Substituindo $v$ por $b^{-s}$:\\
					$z = (b^{s}b^{-s})^{e} b^{r} \modd p$\\
					Percebemos que o termo $[b^{s} \ b^{-s}]^{e}$ é igual
					a $[b^{s - s}]^{e}$, ou seja $[b^{0}]^{e} = 1$:\\
					$z = b^r \modd p$\\
					Como sabemos que $u = b^r\modd p$\\
					$z = u$\\
					$\QED$
				\end{center}	
			\newpage	
			\item[2 -]
				O algoritmo de Assinatura envolve:\\
					\begin{itemize}
						\item[1.]
							Escolha de inteiro aleatório $r$ : $1 \leq r \leq q-1$
						\item[2.]
							Cálculo de $u = b^r \modd p$, $e = h(x||u)$ e
							$y = (s*e + r) \modd q$
						 \item[3.]
						 	A assinatura é $(y, e)$		
					\end{itemize}
				O algoritmo de Verificação envolve:\\
					\begin{itemize}
						\item[1.]
							Autenticar as informações públicas $p, q, b, v$ utilizando
							a assinatura $A_T()$ da autoridade idônea $T$
						\item[2.]
							Cálculo de $z = b^yv^e \modd p$, $e' = h(x||z)$
						\item[3.]
							Verificar se $e = e'$	
					\end{itemize}		
				Percebemos que a \textbf{Assinatura} é mais \textbf{rápida} do que a
				verificação, pois são executadas menos multiplicações (operação mais
				custosa).  Escolher um inteiro aleatório é relativamente rápido, assim
				como autenticar as informações públicas. Ambos os algoritmos devem
				calcular $h()$, porém a assinatura deve calcular apenas uma 
				exponencial ($b^r$) enquanto a verificação deve fazer duas
				exponenciais ($b^y$ e $v^e$) sendo que os expoentes são de
				ordens parecidas.
			\item[3 -]
				Para falsificar uma assinatura Schnorr sobre um texto x a falsa
				Alice deverá escolher $r$ a ser usado e calcular $u$ e $e$,
				então a falsa Alice tenta adivinhar o $y$ e envia $(y, e)$ para Beto.
				Como ele pensa que recebeu os parâmetros da Alice verdadeira,
				irá calcular $z = b^yv^e \modd p$, $e' = h(x||z)$ utilizando
				os parâmetros públicos da Alice verdadeira e como o $y$ foi
				adivinhado pela falsa Alice, Beto chegará em $e = e'$ validando
				a falsa Alice.
			\item[4 -]
				O sucesso de tal falsificação é dado pela probabilidade da falsa
				Alice acertar o parâmetro $y$. Porém como $e$ é calculado
				a partir de uma função de hash $h(x||u)$ devido às colisões de
				$h()$ existe mais de um valor de $u$ que resulta num mesmo $e$,
				sendo assim Alice falsa pode mapear os valores de $r$ para 
				os quais $h(x||u)$ resultam num mesmo $e$. A partir disso o
				trabalho de descobrir os valores de $y$ relacionados a $r$
				são facilitados, pois falsa Alice possui uma nova informação
				sobre como $y$ é gerado.
			\item[5 -]
				Os parâmetros de segurança são:
				\begin{itemize}
					\item[]
						$r$, parâmetro NONCE, protegido pelo PLD quando se
						calcula $u = b^r \modd p$\\
					\item[]
						$s$, protegida pelo PLD na operação $v = b^{-s} \modd p$	
				\end{itemize}	
				O conhecimento desses parâmetros permite uma falsificação, uma
				vez que conhecendo $r$ pode-se calcular $u$, com $u$ calcula-se
				$e = h(x||u)$	 e com $e$ e $r$, conhecendo-se $s$, calcula-se
				$y = (s*e + r) \modd q$	conseguindo falsificar a assinatura $(y, e)$.	
			\item[6 -]
				Para $p = 17$, $q = 8$, $g = 7$, $x = 12$, $s = 9$, $r = 6$\\
				Calcular	 $b$, $v$, $u$, Schnorr $(y, e)$ sobre $x$
				\begin{itemize}
					\item[]
						Cálculo de $b$:\\
						$b = g^{(p-1)/q} \modd p$\\
						$b = 7^{(17-1)/8} \modd 17$\\
						$\mathbf{b = 15}$
					\item[]
						Cálculo de $v$:\\
						$v = b^{-s} \modd p$\\
						$v = 15^{-9} \modd 17$\\
						$\mathbf{v = 8}$
					\item[]
						Cálculo de $u$:\\
						$u = b^r \modd p$\\
						$u = 15^6 \modd 17$\\
						$\mathbf{u = 13}$
					\item[]
						Cálculo de $e$:\\
						$e = h(x||u)$\\
						$e = h(1213)$\\
						$\mathbf{e = 12}$	
					\item[]
						Cálculo de $y$:\\
						$y = (s*e + r) \modd q$\\
						$y = (9*12 + 6) \modd 8$\\
						$\mathbf{y = 2}$						
				\end{itemize}	
			Portanto Schnorr $(y, e) = (2, 12)$	 
			\item[7 -]
				Cálculo de $z = b^y v^e  \modd p$ e $e' = h(x||z)$:\\
				\begin{itemize}
					\item[]
						Para $z$ temos:\\
						$z = 15^2 8^{12}  \modd 17$\\
						$\mathbf{z = 13}$
					\item[]
						Para $e'$ temos:\\
						$e' = h(12||13)$\\
						$e' = h(1213)$\\
						$\mathbf{e' = 12}$	
				\end{itemize}		
				Desta forma verificamos que $e = e' = 12$.									
		\end{itemize}
		\begin{center}
			FIM FIM FIM FIM
		\end{center}
\end{document}