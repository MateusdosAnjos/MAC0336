\documentclass[12pt]{article}

%These tell TeX which packages to use.
\usepackage{array,epsfig}
\usepackage{amsmath}
\usepackage{amsfonts}
\usepackage{amssymb}
\usepackage{amsxtra}
\usepackage{amsthm}
\usepackage{mathrsfs}
\usepackage{color}
\usepackage[utf8]{inputenc}



\newcommand{\lra}{\longrightarrow}
\newcommand{\ra}{\rightarrow}
\newcommand{\surj}{\twoheadrightarrow}
\newcommand{\graph}{\mathrm{graph}}
\newcommand{\bb}[1]{\mathbb{#1}}
\newcommand{\Z}{\bb{Z}}
\newcommand{\Q}{\bb{Q}}
\newcommand{\R}{\bb{R}}
\newcommand{\C}{\bb{C}}
\newcommand{\N}{\bb{N}}
\newcommand{\M}{\mathbf{M}}
\newcommand{\m}{\mathbf{m}}
\newcommand{\MM}{\mathscr{M}}
\newcommand{\HH}{\mathscr{H}}
\newcommand{\Om}{\Omega}
\newcommand{\Ho}{\in\HH(\Om)}
\newcommand{\bd}{\partial}
\newcommand{\del}{\partial}
\newcommand{\bardel}{\overline\partial}
\newcommand{\textdf}[1]{\textbf{\textsf{#1}}\index{#1}}
\newcommand{\img}{\mathrm{img}}
\newcommand{\ip}[2]{\left\langle{#1},{#2}\right\rangle}
\newcommand{\inter}[1]{\mathrm{int}{#1}}
\newcommand{\exter}[1]{\mathrm{ext}{#1}}
\newcommand{\cl}[1]{\mathrm{cl}{#1}}
\newcommand{\ds}{\displaystyle}
\newcommand{\vol}{\mathrm{vol}}
\newcommand{\cnt}{\mathrm{ct}}
\newcommand{\osc}{\mathrm{osc}}
\newcommand{\LL}{\mathbf{L}}
\newcommand{\UU}{\mathbf{U}}
\newcommand{\support}{\mathrm{support}}
\newcommand{\AND}{\;\wedge\;}
\newcommand{\OR}{\;\vee\;}
\newcommand{\Oset}{\varnothing}
\newcommand{\st}{\ni}
\newcommand{\wh}{\widehat}
\newcommand\ceil[1]{\lceil#1\rceil}
\newcommand{\newqed}{{\hfill\color{black}\ensuremath{\blacksquare}}}

%Pagination stuff.
\setlength{\topmargin}{-.3 in}
\setlength{\oddsidemargin}{0in}
\setlength{\evensidemargin}{0in}
\setlength{\textheight}{9.in}
\setlength{\textwidth}{6.5in}
\pagestyle{empty}



\begin{document}

	\begin{center}
		{\Large MAC0336/5723 Criptografia para Segurança de Dados\\
		Lista 3}\\
		\textbf{Mateus Agostinho dos Anjos}\\
		NUSP: 9298191
	\end{center}

	\vspace{0.4 cm}
	
	\subsection*{Exercício 1.}
		Dado os passos de 1 a 6 na página 182 temos:
		\begin{itemize}			
			\item[1 -]
				No passo 3 Beto escolhe um $0 < x < n$ e envia $a$ para
				Alice, tal que $x^2 mod \ n = a$. No passo 4 Alice calcula as 
				quatro raízes quadradas de $a \ mod \ n$ e envia uma delas 
				para Beto. \textbf{Ele ganha} caso \textbf{não receber} $x$ ou 
				$x - n$ (passo 5).\\
				A justificativa do porque Beto ganha e Alice aceita é explicada
				no passo 6. Se Beto receber outra raiz quadrada $y$ ou
				$n - y$ ele consegue fatorar $n$ com facilidade calculando
				$mdc(x+y, n) = p$ e envia a fatoração de $n$ para Alice, que
				aceita a vitória de Beto.
			\item[2 -]
				Alice ganha o jogo caso enviar para Beto $x$ ou $n - x$, pois
				com essas informações ele não consegue calcular a fatoração
				de $n$ e assim provar que ganhou. (Descrito no passo 5)
			\item[3 -]		
				Beto rejeita o caso de $y = 0$, pois caso $y = 0$ fosse válido
				Beto conseguiria calcular a fatoração de $n$ apenas com
				o $x$ escolhido por ele, uma vez que $mdc(x+y, n) = p$ ou $q$ e 
				$n = pq$. Com $y = 0$ teríamos $mdc(x+0, n) = p$ ou $q$, ou
				seja $mdc(x, n) = p$ ou $q$, sendo que Beto conhece $x$ e $n$.
				Como não é possível obter $p$ ou $q$ apenas com $mdc(x, n)$,
				$y = 0$ daria a vitória sempre para Alice.
			\item[4 -]
				$p = 3$, $q = 7$, $x = 4$, $a = ?$, $y = ?$, $mdc(x + y, n) = ?$\\
				$n = pq = 21$\\
				$x^2 \ mod \ n = a$, portanto $4^2 \ mod \ 21  = a$ então $a =  16$
				\newline
				\begin{center}
					Cálculo das raízes:\\
					$x_1 = a^{\frac{p+1}{4}} mod \ p$	e 
					$x_2 = a^{\frac{q+1}{4}} mod \ q$,
					sendo assim temos:\\
					$x_1 = 16^{\frac{3+1}{4}} mod \ 3$, $x_1 = 1$\\
					$x_2 = 16^{\frac{7+1}{4}} mod \ 7$, $x_2 = 4$\\
					Utilizando o Teorema Chinês do resto calcula-se $x_0$
					solução do sistema:
					$ \begin{cases} 
							x_0 = x_1 \ mod \ p \\ x_0 = x_2 \ mod \ q 
						\end{cases}
					$\\
					Simplificando temos:\\
					$x_0 = (x_2pp^{-1} + x_1qq^{-1}) \ mod \ pq$\\
					Calculamos $p^{-1}$ e $q^{-1}$ utilizando o algoritmo
					de Euclides estendido, chegando em:\\
					$p^{-1} = 5 $ e $q^{-1} = 1 $\\
					Portanto: $x_0 = (4*3*5 + 1*7*1) \ mod \ 21$\\
					$x_0 = 4$\\
					Agora para o cálculo das outras 3 raízes temos:\\
					$x_0^{'} = (x_2pp^{-1} - x_1qq^{-1}) \ mod \ pq$, $(pq - x_0)$,
					$(pq - x_0^{'})$\\
					$x_0^{'} = (4*3*5 - 1*7*1) \ mod \ 21 = 11$\\
					$x_0^{''} = 21 - 4 = 17$\\
					$x_0^{'''} = 21 - 11 = 10$\\
					Pegando $y = 11$ temos $mdc(4+11, 21) = mdc(15, 21) = 3 = p$				
				\end{center} 	
				Terminado temos: 
				$p = 3$, $q = 7$, $x = 4$, $\mathbf{a = 16, y = 11, mdc(x + y, n) = 3}$,
				$n = 21$
		\end{itemize}
	\subsection*{Exercício 2.}

\end{document}