\documentclass[12pt]{article}

%These tell TeX which packages to use.
\usepackage{amsxtra}
\usepackage[utf8]{inputenc}

%New commands
\newcommand\ceil[1]{\lceil#1\rceil}
\newcommand{\newqed}{{\hfill\color{black}\ensuremath{\blacksquare}}}

%Pagination stuff.
\setlength{\topmargin}{-.3 in}
\setlength{\oddsidemargin}{0in}
\setlength{\evensidemargin}{0in}
\setlength{\textheight}{9.in}
\setlength{\textwidth}{6.5in}
\pagestyle{empty}

\begin{document}

	\begin{center}
		{\Large MAC0336/5723 Criptografia para Segurança de Dados\\
		Lista 3}\\
		\textbf{Mateus Agostinho dos Anjos}\\
		NUSP: 9298191
	\end{center}

	\vspace{0.4 cm}
	
	\subsection*{Exercício 1.}
		Dado os passos de 1 a 6 na página 182 temos:
		\begin{itemize}			
			\item[1 -]
				No passo 3 Beto escolhe um $0 < x < n$ e envia $a$ para
				Alice, tal que $x^2 mod \ n = a$. No passo 4 Alice calcula as 
				quatro raízes quadradas de $a \ mod \ n$ e envia uma delas 
				para Beto. \textbf{Ele ganha} caso \textbf{não receber} $x$ ou 
				$x - n$ (passo 5).\\
				A justificativa do porque Beto ganha e Alice aceita é explicada
				no passo 6. Se Beto receber outra raiz quadrada $y$ ou
				$n - y$ ele consegue fatorar $n$ com facilidade calculando
				$mdc(x+y, n) = p$ e envia a fatoração de $n$ para Alice, que
				aceita a vitória de Beto.
			\item[2 -]
				Alice ganha o jogo caso enviar para Beto $x$ ou $n - x$, pois
				com essas informações ele não consegue calcular a fatoração
				de $n$ e assim provar que ganhou. (Descrito no passo 5)
			\item[3 -]
				Sabemos que $n = pq$.\\		
				Assumindo que $y = 0$ fosse uma raiz válida que permitisse
				a Beto obter a fatoração de $n$, uma vez que $y \neq x$ e
				$y \neq n - x$.\\
				Beto conseguiria calcular a fatoração de $n$ apenas com
				o $x$ escolhido por ele, uma vez que $mdc(x+y, n) = p$ ou $q$, 
				com $y = 0$ teríamos $mdc(x+0, n) = p$ ou $q$, ou
				seja $mdc(x, n) = p$ ou $q$, sendo que Beto conhece $x$ e $n$.
				Como não é possível obter $p$ ou $q$ apenas com $mdc(x, n)$,
				$y = 0$ é inválido, caracterizando trapaça de Alice que sempre daria a
				vitória a ela, pelo fato de Beto não conseguir provar que venceu 
				(obter a fatoração de $n$).
			\item[4 -]
				$p = 3$, $q = 7$, $x = 4$, $a = ?$, $y = ?$, $mdc(x + y, n) = ?$\\
				$n = pq = 21$\\
				$x^2 \ mod \ n = a$, portanto $4^2 \ mod \ 21  = a$ então $a =  16$
				\newline
				\begin{center}
					Cálculo das raízes:\\
					$x_1 = a^{\frac{p+1}{4}} mod \ p$	e 
					$x_2 = a^{\frac{q+1}{4}} mod \ q$,
					sendo assim temos:\\
					$x_1 = 16^{\frac{3+1}{4}} mod \ 3$, $x_1 = 1$\\
					$x_2 = 16^{\frac{7+1}{4}} mod \ 7$, $x_2 = 4$\\
					Utilizando o Teorema Chinês do resto calcula-se $x_0$
					solução do sistema:
					$ \begin{cases} 
							x_0 = x_1 \ mod \ p \\
							x_0 = x_2 \ mod \ q 
						\end{cases}
					$\\
					Simplificando temos:\\
					$x_0 = (x_2pp^{-1} + x_1qq^{-1}) \ mod \ pq$\\
					Calculamos $p^{-1}$ e $q^{-1}$ utilizando o algoritmo
					de Euclides estendido, chegando em:\\
					$p^{-1} = 5 $ e $q^{-1} = 1 $\\
					Portanto: $x_0 = (4*3*5 + 1*7*1) \ mod \ 21$\\
					$x_0 = 4$\\
					Agora para o cálculo das outras 3 raízes temos:\\
					$x_0^{'} = (x_2pp^{-1} - x_1qq^{-1}) \ mod \ pq$, $(pq - x_0)$,
					$(pq - x_0^{'})$\\
					$x_0^{'} = (4*3*5 - 1*7*1) \ mod \ 21 = 11$\\
					$x_0^{''} = 21 - 4 = 17$\\
					$x_0^{'''} = 21 - 11 = 10$\\
					Pegando $y = 11$ temos $mdc(4+11, 21) = mdc(15, 21) = 3 = p$				
				\end{center} 	
				Terminado temos: 
				$p = 3$, $q = 7$, $x = 4$, $\mathbf{a = 16, y = 11, mdc(x + y, n) = 3}$,
				$n = 21$
		\end{itemize}
	\subsection*{Exercício 2.}
		\begin{itemize}
			\item[1 -]
				Sabemos que o testemunho $x = r^2 \ mod \ n $\\
				Sabemos que $v = s^2 \ mod \ n$\\
				$y$ é autêntico, portanto vale que:
				\begin{center}								
					$ \begin{cases} 
					y = r \ mod \ n, & e = 0 \\ 
					y = rs \ mod \ n, & e = 1
					\end{cases}
					$
				\end{center}
				Para $e = 0$:\\
				$xv^{e} \ mod \ n = xv^{0} \ mod \ n = x \ mod \ n$\\
				\begin{center} 
					$y^2 = r^2 \ mod \ n$\\
					$y^2 = x \ mod \ n$\\
				\end{center}
				Vemos que:
				$ \begin{cases} 
					xv^{e} \ mod \ n =  x \ mod \ n \\ y^2 = x \ mod \ n
					\end{cases}
				$\\
				Concluindo que, para $e = 0$ e $y$ autêntico, vale que $y^2 = xv^{e} \
				mod \ n$\\
				\newline
				Para $e = 1$:\\
				$xv^{e} \ mod \ n = xv \ mod \ n$\\
				\begin{center} 
					$y^2 = (rs)^2 \ mod \ n$\\
					$y^2 = r^2s^2 \ mod \ n$\\
					$y^2 = (r^2 \ mod \ n) \ (s^2 \ mod \ n)$\\
					$y^2 = (x \ mod \ n) \ (v \ mod \ n)$\\
					$y^2 = xv \ mod \ n$\\					
				\end{center}
				Concluindo que, para $e = 1$ e $y$ autêntico, vale que $y^2 = xv^{e} \
				mod \ n$
			\item[2 -]
				Para o protocolo de identificação Feige, Fiet e Shamir os parâmetros
				de segurança são:
					\begin{itemize}
						\item[•]
							O inteiro $\mathbf{s}$ relativamente primo a $n$, escolhido por 
							Alice, protegido pelo problema da fatoração de $n$, sendo 
							computacionalmente difícil calcular $s$ conhecendo-se apenas
							$v$ e $n$. O conhecimento de $\mathbf{s}$ facilitaria a
							personificação de Alice (no passo do envio de $y = rs$ para
							Beto) por algum mal intencionado.
						\item[•]
							O inteiro $\mathbf{r}$, protegido pela fatoração de $n$.
							Conhecendo-se $\mathbf{r}$ algum mal intencionado
							poderia enviar o testemunho $x$ para Beto, pois 
							$x = r^2 \ mod \ n$ com $n$ conhecido e personificar
							Alice.
						\item[•]
							O desafio $e$ pode ser considerado um parâmetro de
							segurança, pois impede o ataque de um espião que mapeou
							todos os pares $x = r^2, y = rs$ a fim de responder
							$y = rs$ no passo 3, já que para $e = 1$ o passo 4
							seria $y^2 = xv = r^2s^2$. Com o desafio $e = 0$, 
							mapear todos os valores não auxilia o espião, já que a
							resposta exige $y = \sqrt{x} \ mod \ n$ e fazer este
							cálculo sem a fatoração de $n$ é computacionalmente
							difícil. Portanto, pode-se dizer que o problema da fatoração
							de $n$ também protege a verificação quando é feito o
							desafio $e$.						
					\end{itemize}	
				Portanto conhecer $\mathbf{s}$ ou $\mathbf{r}$ facilita para um
				mal feitor personificar Alice, porém é necessário ter o conhecimento
				dos dois parâmetros para obter total sucesso na personificação.
			\item[3 -]
				O protocolo Feige, Fiet e Shamir é do tipo Zero Knowledge, pois
				permite a Beto verificar que é Alice verdadeira que manda as
				mensagens sem obter conhecimento sobre nenhuma informação
				privada dela, ou seja Beto não precisa saber qual a chave 
				$s$ utilizada por Alice para efetuar a verificação.
			\item[4 -]
				Para $t = 1$, $p = 3$, $q = 7$, $s = 17$, $r = 13$\\
				Temos:
				\begin{center}
					Cálculo de $n$:\\
					$n = pq$\\
					$n = 3*7$\\
					$\mathbf{n = 21}$\\
				\end{center}
				\begin{center}
					Cálculo de $v$:\\					
					$v = s^2 \ mod \ n$\\
					$v = 17^2 \ mod \ 21$\\
					$\mathbf{v = 16}$\\
				\end{center}
				\begin{center}
					Cálculo de $x$:\\					
					$x = r^2 \ mod \ n$\\
					$x = 13^2 \ mod \ 21$\\
					$\mathbf{x = 1}$\\
				\end{center}
				\newpage
				Para $e = 0$:
				\begin{center}
					Cálculo de $y$:\\					
					$y = rs^{e} \ mod \ n$\\
					$y = 13*17^{0} \ mod \ 21$\\					
					$\mathbf{y = 13}$\\
				\end{center}		
				\begin{center}
					Cálculo de $y^2$:\\					
					$y^2 \ mod \ n$\\
					$13^2 \ mod \ 21$\\
					$\mathbf{y^2 = 1}$\\
				\end{center}
				\begin{center}
					Cálculo de $xv^e \ mod \ n$:\\					
					$1*16^0 \ mod \ 21$\\
					$\mathbf{xv^e \ mod \ n = 1}$\\
				\end{center}
				Verificando, portanto, que $y^2 = xv^e \ mod \ n$ para $e = 0$\\
				Para $e = 1$:
				\begin{center}
					Cálculo de $y$:\\					
					$y = rs^{e} \ mod \ n$\\
					$y = 13*17^{1} \ mod \ 21$\\					
					$\mathbf{y = 11}$\\
				\end{center}		
				\begin{center}
					Cálculo de $y^2$:\\					
					$y^2 \ mod \ n$\\
					$11^2 \ mod \ 21$\\
					$\mathbf{y^2 = 16}$\\
				\end{center}
				\begin{center}
					Cálculo de $xv^e \ mod \ n$:\\					
					$1*16^1 \ mod \ 21$\\
					$\mathbf{xv^e \ mod \ n = 16}$\\
				\end{center}
				Verificando, portanto, que $y^2 = xv^e \ mod \ n$ para $e = 1$	
		\end{itemize}				
	\subsection*{Exercício 3.}
																					

\end{document}