\documentclass[12pt]{article}

%These tell TeX which packages to use.
\usepackage{array,epsfig}
\usepackage{amsmath}
\usepackage{amsfonts}
\usepackage{amssymb}
\usepackage{amsxtra}
\usepackage{amsthm}
\usepackage{mathrsfs}
\usepackage{color}
\usepackage[utf8]{inputenc}



\newcommand{\lra}{\longrightarrow}
\newcommand{\ra}{\rightarrow}
\newcommand{\surj}{\twoheadrightarrow}
\newcommand{\graph}{\mathrm{graph}}
\newcommand{\bb}[1]{\mathbb{#1}}
\newcommand{\Z}{\bb{Z}}
\newcommand{\Q}{\bb{Q}}
\newcommand{\R}{\bb{R}}
\newcommand{\C}{\bb{C}}
\newcommand{\N}{\bb{N}}
\newcommand{\M}{\mathbf{M}}
\newcommand{\m}{\mathbf{m}}
\newcommand{\MM}{\mathscr{M}}
\newcommand{\HH}{\mathscr{H}}
\newcommand{\Om}{\Omega}
\newcommand{\Ho}{\in\HH(\Om)}
\newcommand{\bd}{\partial}
\newcommand{\del}{\partial}
\newcommand{\bardel}{\overline\partial}
\newcommand{\textdf}[1]{\textbf{\textsf{#1}}\index{#1}}
\newcommand{\img}{\mathrm{img}}
\newcommand{\ip}[2]{\left\langle{#1},{#2}\right\rangle}
\newcommand{\inter}[1]{\mathrm{int}{#1}}
\newcommand{\exter}[1]{\mathrm{ext}{#1}}
\newcommand{\cl}[1]{\mathrm{cl}{#1}}
\newcommand{\ds}{\displaystyle}
\newcommand{\vol}{\mathrm{vol}}
\newcommand{\cnt}{\mathrm{ct}}
\newcommand{\osc}{\mathrm{osc}}
\newcommand{\LL}{\mathbf{L}}
\newcommand{\UU}{\mathbf{U}}
\newcommand{\support}{\mathrm{support}}
\newcommand{\AND}{\;\wedge\;}
\newcommand{\OR}{\;\vee\;}
\newcommand{\Oset}{\varnothing}
\newcommand{\st}{\ni}
\newcommand{\wh}{\widehat}
\newcommand\ceil[1]{\lceil#1\rceil}
\newcommand{\newqed}{{\hfill\color{black}\ensuremath{\blacksquare}}}

%Pagination stuff.
\setlength{\topmargin}{-.3 in}
\setlength{\oddsidemargin}{0in}
\setlength{\evensidemargin}{0in}
\setlength{\textheight}{9.in}
\setlength{\textwidth}{6.5in}
\pagestyle{empty}



\begin{document}


\begin{center}
{\Large MAC0336/5723 Criptografia para Segurança de Dados\\
Lista 2}\\
\textbf{Mateus Agostinho dos Anjos}\\
NUSP: 9298191
\end{center}

\vspace{0.4 cm}

\subsection*{Exercício 1.}
	\begin{itemize}
		\item[1. ]
			Temos como entrada:	 $n$ e $\Phi(n)$
		\item[2. ]	
			Sabemos que no algoritmo do RSA: $n = p * q$ e $\Phi(n) = (p-1)*(q-1)$
		\item[3. ]	
			Queremos descobrir $p$ e $q$	para fatorar $n$
	\end{itemize}
	Podemos manipular essas equações da seguinte forma:
	$$p - 1 = \frac{\Phi(n)}{(q - 1)}  \ \ , \ \  (q - 1) > 0$$
	\begin{equation}
		\label{eqn:I} \tag{I}
			p = \frac{\Phi(n)}{q - 1} + 1 
	\end{equation}
	Substituindo \eqref{eqn:I} em $n = p * q$ temos:
	$$ n = \Big(\frac{\Phi(n)}{q - 1} + 1\Big) * q $$
	$$ n = (\Phi(n) + q - 1) * q$$
	$$ n = \Phi(n)*q + q^2 - q$$
	\begin{equation}
		\label{eqn:II} \tag{II}
			 n = q^2 + (\Phi(n) - 1) * q
	\end{equation}
	\textbf{Note que}, se isolássemos $q$ em \eqref{eqn:I} chegaríamos em:
	$$n = p^2 + (\Phi(n) - 1) * p$$
	Portanto as soluções de $p$ e $q$ são simétricas.\\
	
	Como temos $n$ e $\Phi(n)$ podemos achar as raízes $r_1$ e $r_2$ de
	\eqref{eqn:II}, sendo que $p = r_1$ e $q = r_2$, uma vez que
	$p$ e $q$ são primos a fatoração de $n$ será $p * q$, portanto temos
	um algoritmo \textit{rápido} (solução de uma equação de segundo grau, podendo
	utilizar $bhaskara$) para encontrar a fatoração de $n$ .
	
	\newqed

\newpage	
\subsection*{Exercício 2.}
	\begin{itemize}
		\item[1. ]
			$n = 21$ e $a = 5$\\
			Fatorando $n-1$ temos: $20 = 2^2*5$\\
			Portanto $t = 2$ e $c = 5$
			Agora calcularemos os módulos
			\begin{align*}
				(5^5)^1 &\equiv 17 \ \mathrm{mod} \ 21 \\
				(5^5)^2 &\equiv 16 \ \mathrm{mod} \ 21 \\
				(5^5)^4 &\equiv 4 \ \mathrm{mod} \ 21
			\end{align*}
			$r_0 = 17$ , $r_1 = 16$ , $r_2 = 4$\\
			Como nenhum $r_x$ é igual a +1 ou -1 temos que o número 21 é composto.\\
			Esta resposta final está correta, uma vez que 21 é divisível por 3 e por 7
			ele possui mais do que os 2 divisores naturais triviais, portanto não é primo.
		\item[2. ]
			$n = 13$ e $a = 2$\\
			Fatorando $n-1$ temos: $12 = 2^2*3$\\
			Portanto $t = 2$ e $c = 3$
			Agora calcularemos os módulos
			\begin{align*}
				(2^3)^1 &\equiv 8 \ \mathrm{mod} \ 13 \\
				(2^3)^2 &\equiv 12 \ \mathrm{mod} \ 13 \\
				(2^3)^4 &\equiv 1 \ \mathrm{mod} \ 13
			\end{align*}
			$r_0 = 8$ , $r_1 = 12 \ (-1)$, $r_2 = 1$\\
			Como $r_2$ é igual 1 e $r_x$ imediatamente anterior a ele é 
			$r_1 = 12 \ (-1)$ o número $n$ é dado como primo.\\
			Esta resposta final está correta, uma vez que 13 é primo.	
	\end{itemize}
\subsection*{Exercício 3.}
	\textbf{Enunciado:}\\
	Demonstre que se $x$ é uma raiz quadrada de $1$ mod $n$ distinto de
	$1$ mod $n$ e de $-1$ mod $n$, então $mdc(x-1, n)$ e $mdc(x+1, n)$
	são ambos divisores não triviais de $n$.
	\newline
	\newline
	\textbf{Demonstração:}\\
	Temos, por suposição, que x é uma raiz quadrada de $1$ mod $n$, portanto:
	\begin{align*}
			x^2 &\equiv 1 \ \mathrm{mod} \ n\\
			x^2 - 1 &\equiv 0 \ \mathrm{mod} \ n\\
			\label{eqn:III}\tag{I}
				(x - 1)(x + 1) &\equiv 0 \ \mathrm{mod} \ n\\
	\end{align*}
	Da equação \eqref{eqn:III} sabemos que $n$ divide o produto de $(x-1)(x+1)$,
	portanto $n$ tem fatores em comum com $(x-1)$ e com $(x+1)$.\\
	$$\frac{(x-1)(x+1)}{n} = i,   \text{para algum} \  i \in \N$$
	Portanto podemos pegar o $mdc(x-1, n)$ bem como o $mdc(x+1, n)$ como 
	divisores  não triviais de $n$.
	
	\newqed
	
\subsection*{Exercício 4.}
	\textbf{Enunciado:}\\
		Demonstre que se $q, r$ são primos distintos, $n = qr$, $0 < a < n$, 
		e se $x, y$ são raízes quadradas de $a \ \mathrm{mod} \ n$ tais que
		 $y \neq x \ \mathrm{mod} \ n$ e $y \neq n-x \ \mathrm{mod} \ n$, 
		 então $mdc(x-y, n) = q$ ou $= r$.
	\newline
	\newline
	\textbf{Demonstração:}\\
		Temos:
		\begin{itemize}
		\item
			$n = qr$
		\item
			$q, r$ são primos distintos.
		\item
			$x^2 \equiv a \ \mathrm{mod} \ n$
		\item
			$y^2 \equiv a \ \mathrm{mod} \ n$
		\item
			$y \neq x \ \mathrm{mod} \ n$
		\item
			$y \neq n-x \ \mathrm{mod} \ n$	
		\end{itemize}
		Operando com o que nos foi dado:
		\begin{align*}
			x^2 - y^2 &\equiv 0 \ \mathrm{mod} \ n\\
			(x+y)(x-y) &\equiv 0 \ \mathrm{mod} \ n\\
			\text{Como} \ y \neq x \ \mathrm{mod} \ &n \ 
			\text{Então} \ (x-y) \neq 0\\
			\text{Como} \ y \neq n-x \ \mathrm{mod} \ &n \ 
			\text{Então} \ (x+y) \neq 0\\
			\label{eqn:IV}\tag{I}
			\text{Sendo assim:} \ \frac{(x+y)(x-y)}{n} &= i, 
			\text{para algum} \  i \in \N\\
			\text{Portanto} \ n \ \text{tem fator(es) em} \ &
			\text{comum com} \ (x+y) \ \text{e} \ (x-y)
		\end{align*}
		Como $n = qr$ e $q, r$ são primos distintos, os únicos divisores de $n$ são:
		$1, q, r, n$.\\ 
		Podemos reescrever \eqref{eqn:IV} substituindo $n$ por $qr$:
		$$\text{Sendo assim:} \ \frac{(x+y)(x-y)}{qr} = i, 
		\text{para algum} \  i \in \N\\$$
		Portanto $mdc(x-y, n) = q$ ou $= r$.
			
		\newqed
\end{document}