\documentclass[12pt]{article}

%These tell TeX which packages to use.
\usepackage{array,epsfig}
\usepackage{amsmath}
\usepackage{amsfonts}
\usepackage{amssymb}
\usepackage{amsxtra}
\usepackage{amsthm}
\usepackage{mathrsfs}
\usepackage{color}
\usepackage[utf8]{inputenc}



\newcommand{\lra}{\longrightarrow}
\newcommand{\ra}{\rightarrow}
\newcommand{\surj}{\twoheadrightarrow}
\newcommand{\graph}{\mathrm{graph}}
\newcommand{\bb}[1]{\mathbb{#1}}
\newcommand{\Z}{\bb{Z}}
\newcommand{\Q}{\bb{Q}}
\newcommand{\R}{\bb{R}}
\newcommand{\C}{\bb{C}}
\newcommand{\N}{\bb{N}}
\newcommand{\M}{\mathbf{M}}
\newcommand{\m}{\mathbf{m}}
\newcommand{\MM}{\mathscr{M}}
\newcommand{\HH}{\mathscr{H}}
\newcommand{\Om}{\Omega}
\newcommand{\Ho}{\in\HH(\Om)}
\newcommand{\bd}{\partial}
\newcommand{\del}{\partial}
\newcommand{\bardel}{\overline\partial}
\newcommand{\textdf}[1]{\textbf{\textsf{#1}}\index{#1}}
\newcommand{\img}{\mathrm{img}}
\newcommand{\ip}[2]{\left\langle{#1},{#2}\right\rangle}
\newcommand{\inter}[1]{\mathrm{int}{#1}}
\newcommand{\exter}[1]{\mathrm{ext}{#1}}
\newcommand{\cl}[1]{\mathrm{cl}{#1}}
\newcommand{\ds}{\displaystyle}
\newcommand{\vol}{\mathrm{vol}}
\newcommand{\cnt}{\mathrm{ct}}
\newcommand{\osc}{\mathrm{osc}}
\newcommand{\LL}{\mathbf{L}}
\newcommand{\UU}{\mathbf{U}}
\newcommand{\support}{\mathrm{support}}
\newcommand{\AND}{\;\wedge\;}
\newcommand{\OR}{\;\vee\;}
\newcommand{\Oset}{\varnothing}
\newcommand{\st}{\ni}
\newcommand{\wh}{\widehat}
\newcommand\ceil[1]{\lceil#1\rceil}
\newcommand{\newqed}{{\hfill\color{black}\ensuremath{\blacksquare}}}

%Pagination stuff.
\setlength{\topmargin}{-.3 in}
\setlength{\oddsidemargin}{0in}
\setlength{\evensidemargin}{0in}
\setlength{\textheight}{9.in}
\setlength{\textwidth}{6.5in}
\pagestyle{empty}



\begin{document}


\begin{center}
{\Large MAC0336/5723 Criptografia para Segurança de Dados\\
Lista 1}\\
\textbf{Mateus Agostinho dos Anjos}\\
NUSP: 9298191
\end{center}

\vspace{0.4 cm}

\subsection*{Exercício 1.}
	\begin{itemize}
		\item[1. ]
			Temos como entrada:	 $n$ e $\Phi(n)$
		\item[2. ]	
			Sabemos que no algoritmo do RSA: $n = p * q$ e $\Phi(n) = (p-1)*(q-1)$
		\item[3. ]	
			Queremos descobrir $p$ e $q$	para fatorar $n$
	\end{itemize}
	Podemos manipular essas equações da seguinte forma:
	$$p - 1 = \frac{\Phi(n)}{(q - 1)}  \ \ , \ \  (q - 1) > 0$$
	\begin{equation}
		\label{eqn:I} \tag{I}
			p = \frac{\Phi(n)}{q - 1} + 1 
	\end{equation}
	Substituindo \eqref{eqn:I} em $n = p * q$ temos:
	$$ n = \Big(\frac{\Phi(n)}{q - 1} + 1\Big) * q $$
	$$ n = (\Phi(n) + q - 1) * q$$
	$$ n = \Phi(n)*q + q^2 - q$$
	\begin{equation}
		\label{eqn:II} \tag{II}
			 n = q^2 + (\Phi(n) - 1) * q
	\end{equation}
	\textbf{Note que}, se isolássemos $q$ em \eqref{eqn:I} chegaríamos em:
	$$n = p^2 + (\Phi(n) - 1) * p$$
	Portanto as soluções de $p$ e $q$ são simétricas.\\
	
	Como temos $n$ e $\Phi(n)$ podemos achar as raízes $r_1$ e $r_2$ de
	\eqref{eqn:II}, sendo que $p = r_1$ e $q = r_2$, uma vez que
	$p$ e $q$ são primos a fatoração de $n$ será $p * q$, portanto temos
	um algoritmo \textit{rápido} (solução de uma equação de segundo grau, podendo
	utilizar $bhaskara$) para encontrar a fatoração de $n$ .
	
	\newqed

\newpage	
\subsection*{Exercício 2.}
	\begin{itemize}
		\item[1. ]
			$n = 21$ e $a = 5$\\
			Fatorando $n-1$ temos: $20 = 2^2*5$
		\item[2. ]	
	\end{itemize}
\subsection*{Exercício 3.}
\subsection*{Exercício 4.}

\end{document}