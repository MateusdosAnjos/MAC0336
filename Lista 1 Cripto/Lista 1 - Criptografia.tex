\documentclass[12pt]{article}

%These tell TeX which packages to use.
\usepackage{array,epsfig}
\usepackage{amsmath}
\usepackage{amsfonts}
\usepackage{amssymb}
\usepackage{amsxtra}
\usepackage{amsthm}
\usepackage{mathrsfs}
\usepackage{color}
\usepackage[utf8]{inputenc}



\newcommand{\lra}{\longrightarrow}
\newcommand{\ra}{\rightarrow}
\newcommand{\surj}{\twoheadrightarrow}
\newcommand{\graph}{\mathrm{graph}}
\newcommand{\bb}[1]{\mathbb{#1}}
\newcommand{\Z}{\bb{Z}}
\newcommand{\Q}{\bb{Q}}
\newcommand{\R}{\bb{R}}
\newcommand{\C}{\bb{C}}
\newcommand{\N}{\bb{N}}
\newcommand{\M}{\mathbf{M}}
\newcommand{\m}{\mathbf{m}}
\newcommand{\MM}{\mathscr{M}}
\newcommand{\HH}{\mathscr{H}}
\newcommand{\Om}{\Omega}
\newcommand{\Ho}{\in\HH(\Om)}
\newcommand{\bd}{\partial}
\newcommand{\del}{\partial}
\newcommand{\bardel}{\overline\partial}
\newcommand{\textdf}[1]{\textbf{\textsf{#1}}\index{#1}}
\newcommand{\img}{\mathrm{img}}
\newcommand{\ip}[2]{\left\langle{#1},{#2}\right\rangle}
\newcommand{\inter}[1]{\mathrm{int}{#1}}
\newcommand{\exter}[1]{\mathrm{ext}{#1}}
\newcommand{\cl}[1]{\mathrm{cl}{#1}}
\newcommand{\ds}{\displaystyle}
\newcommand{\vol}{\mathrm{vol}}
\newcommand{\cnt}{\mathrm{ct}}
\newcommand{\osc}{\mathrm{osc}}
\newcommand{\LL}{\mathbf{L}}
\newcommand{\UU}{\mathbf{U}}
\newcommand{\support}{\mathrm{support}}
\newcommand{\AND}{\;\wedge\;}
\newcommand{\OR}{\;\vee\;}
\newcommand{\Oset}{\varnothing}
\newcommand{\st}{\ni}
\newcommand{\wh}{\widehat}
\newcommand\ceil[1]{\lceil#1\rceil}

%Pagination stuff.
\setlength{\topmargin}{-.3 in}
\setlength{\oddsidemargin}{0in}
\setlength{\evensidemargin}{0in}
\setlength{\textheight}{9.in}
\setlength{\textwidth}{6.5in}
\pagestyle{empty}



\begin{document}


\begin{center}
{\Large MAC0336/5723 Criptografia para Segurança de Dados\\
Lista 1}\\
\textbf{Mateus Agostinho dos Anjos}\\
NUSP: 9298191
\end{center}

\vspace{0.2 cm}

\subsection*{Exercício 1.}
	Sabemos que:\\
	Dada $n$ informações $X = \{ x_1, x_2, \dots, x_n\}$ ocorrendo com as respectivas probabilidades
	$p(x_1), p(x_2), \dots, p(x_n)$  a Entropia é definida pela fórmula:\\
	\[E(X) = \sum_{j = 1} ^ {n} p(x_j)\log_2(\frac{1}{p(x_j)})\]
\begin{enumerate}	 
	\item
	Dados do enunciado:\\
	$p(x_1) = 1/16\\ p(x_2) = 1/4\\p(x_3) = 1/16\\p(x_4) = 1/4\\p(x_5) = 1/4\\ p(x_6) = 1/16\\p(x_7) = 1/16\\$
	Aplicando a fórmula aos dados do enunciado temos:
	\[4 *\frac{1}{16}\log_2(\frac{1}{(1/16)}) + 3 * \frac{1}{4}\log_2(\frac{1}{(1/4)})\]
	\[\frac{1}{4}\log_2(\frac{1}{(1/16)}) + \frac{3}{4}\log_2(\frac{1}{(1/4)})\]
	\[\frac{1}{4}\log_2(16) + \frac{3}{4}\log_2(4)\]
	\[\frac{1}{4} * 4 + \frac{3}{4} * 2\]
	\[1 + \frac{3}{2}\]
	\[\frac{5}{2}\]
	\Large Resposta: Entropia de $X = \frac{5}{2}$
	\newpage
	\normalsize
	\item
	Queremos demonstrar que, para $j = 1, 2, \dots, n$, $max_j\ceil{\log_2[\frac{1}{p(x_j)}]}$  representa
	o comprimento suficiente de bits para codificar cada um dos $x_j : j = 1, 2, \dots, n$\\\\
	Veja que para j = 1 temos:\[max_1\ceil{\log_2[\frac{1}{p(x_1)}]} = 0\]
	Para j = 2 devemos selecionar o máximo dentre os 2 valores\\
	$$\sum_{j = 1} ^ {2} p(x_j)\log_2(\frac{1}{p(x_j)}) \leq \sum_{j = 1} ^ {2} p(x_j) max_j({\log_2[\frac{1}{p(x_j)}]})$$
	Do mesmo modo, para j = n, temos:
	$$\sum_{j = 1} ^ {n} p(x_j)\log_2(\frac{1}{p(x_j)}) \leq \sum_{j = 1} ^ {n} p(x_j)  max_j({\log_2[\frac{1}{p(x_j)}]})$$
	Como o termo $max_j({\log_2[\frac{1}{p(x_j)}]})$ é igual em todas as parcelas do somatório à direita da inegualdade, podemos 
	colocá-lo em evidência, ficando com:
	$$\sum_{j = 1} ^ {n} p(x_j)\log_2(\frac{1}{p(x_j)}) \leq \Big(\underbrace{\sum_{j = 1} ^ {n} p(x_j)\Big)}_\text{1}  max_j({\log_2[\frac{1}{p(x_j)}]})$$
	$$\underbrace{\sum_{j = 1} ^ {n} p(x_j)\log_2(\frac{1}{p(x_j)})}_\text{Entropia} \leq 1 * max_j({\log_2[\frac{1}{p(x_j)}]})$$
	\Large Como o valor da entropia é menor ou igual a $max_j\ceil{\log_2[\frac{1}{p(x_j)}]}$ ele é um valor
	suficientemente grande de bits para decodificar os n valores propostos em $X$.
	\normalsize
	$\blacksquare$
	\newpage
	\item
	Do enunciado temos que:
	\subitem
		I. A função $log_2{()}$ é estritamente côncava.
	\subitem
		II. Se $f : \R \to \R$ é uma função contínua estritamente côncava no intervalo $ I $, então
		$$\sum_{i = 1} ^ {n} a_i f(y_i) \leq f(\sum_{i = 1} ^ {n} a_iy_i)$$ 
		\subitem
		onde, para $1 \leq i \leq n : a_i \in
		\R, a_i > 0$ e $\sum_{i =  1} ^ {n} a_i = 1$
		
	Temos que as probabilidades $p(x_i) \in \R, p(x_i) > 0$ de $X = \{x_1, x_2, \dots, x_n\}$ somam 1, portanto
	 $a_i$, definido em II, será $p(x_i)$.\\
	 Seja $f(y) = log_2(y)$ temos que $f(y)$ é uma função estritamente côncava no intervalo $[0, 1]$,
	 pois, como visto em I, a função $log_2()$ é estritamente côncava.\\
	 Portanto, vale que:
	 $${\sum_{i = 1} ^ {n} p(x_i) log_2 y_i} \leq log_2({\sum_{i = 1} ^ {n} p(x_i)y_i)}$$
	 Tomando $y_i$ como $\frac {1}{p(x_i)}$ temos:
	 $${\sum_{i = 1} ^ {n} p(x_i) log_2 (\frac{1}{p(x_i)})} \leq log_2({\sum_{i = 1} ^ {n} \frac{p(x_i)}{p(x_i)})}$$
	 $${\sum_{i = 1} ^ {n} p(x_i) log_2 (\frac{1}{p(x_i)})} \leq log_2({\sum_{i = 1} ^ {n} 1)}$$
	 $${\underbrace{\sum_{i = 1} ^ {n} p(x_i) log_2 (\frac{1}{p(x_i)})}_\text{Entropia de X} \leq log_2(n)}$$
	 $\hfill\blacksquare$
	 
	 \item
	 	Provamos que a entropia máxima é de $log_2 n$, portanto 
	 	$$\sum_{i = i} ^ {n} p(x_i)log_2(\frac{1}{p(x_i)}) \leq log_2 n$$
	 	Trabalhando com $log_2 n$ para achar cadidatos a $p(x_i)$:
	 	$$log_2 n = log_2 (\frac{1}{1/n})$$
	 	Veja que $p(x_i) =  \frac{1}{n}$ é um cadidato, verificaremos se é suficiente:
	 	$$\sum_{i = i} ^ {n} \frac{1}{n}log_2(\frac{1}{1/n}) = n * \frac{1}{n}log_2(\frac{1}{1/n}) = log_2 n $$
	 	Nosso cadidato nos levou a um valor máximo de entropia (valor provado no item 3). 
	 	Portanto, um conjunto $X = \{x_1, x_2, \dots, x_n\}$ tal que $\forall x_i \in X, p(x_i) = \frac{1}{n}$,
	 	nos leva a uma entropia de valor máximo.
	
	 

\end{enumerate}

\subsection*{Exercício 2.}
	\begin{center}
	Neste exercício mostraremos os passos das divisões da seguinte forma:\\
			1. $Dividendo =$ Polinômio a ser dividido no momento\\
			2. $Divisor =$ Polinômio que está dividindo\\
			3. $Quociente =$ Valor do quociente no término desta etapa\\
			4. $(Alter. \ no \ Quociente \ da \ etapa \ anterior) \times Divisor =$ Polinômio a ser subtraído do Dividendo\\
			5. $Resto =$ Polinômio resultante da operação (linha 1 - linha 4)\\
	\end{center}		
\begin{enumerate}
	\item
		$$(B2)_{16} = (10110010)_2$$ 
		$$(15)_{16} = (10101)_2$$
		Polinômio $s(x) = x^7 + x^5 + x^4 + x $\\
		Polinômio $t(x) = x^4 + x^2 + 1 $\\
		$$u(x) = s(x) \times t(x) = (( x^7 + x^5 + x^4 + x ) \times (x^4 + x^2 + 1)) mod 2 = x^{11} + x^8 + x^6 + x^4 + x^3 + x$$
		$$u(x) = x^{11} + x^8 + x^6 + x^4 + x^3 + x$$ 
		$$m(x) = x^8 + x^4 + x^3 + x + 1$$
		Etapas da Divisão ($\oplus$ definido como XOR):
		\subitem 
			$$Dividendo = x^{11} + x^8 + x^6 + x^4 + x^3 + x$$
			$$Divisor = x^8 + x^4 + x^3 + x + 1 $$
			$$Quociente = x^3$$
			$$x^3 \times Divisor = x^{11} + x^7 + x^6 + x^4 + x^3$$
			$$Resto = x^8 + x^7 + x$$	
		\subitem 
			$$Dividendo = x^8 + x^7 + x$$
			$$Divisor = x^8 + x^4 + x^3 + x + 1 $$
			$$Quociente = x^3 + 1$$
			$$1 \times Divisor = x^8 + x^4 + x^3 + x + 1$$
			$$Resto = x^7 + x^4 + x^3 + 1$$
		Resultados:
			$$Quociente: \ q(x) = x^3 + 1$$
			$$Resto: \ r(x) = x^7 + x^4 + x^3 + 1$$		
	\item
		Calculo de $r^{-1}(x)$ mod $m(x)$:\\
		Do algoritmo de Euclides Estendido temos: $X \times a + Y \times b = mdc(X, Y)$\\
		Adaptando para o nosso exercício ficamos com: $r(x)\times a + m(x) \times b = mdc(r(x), m(x))$, 
		verificaremos que $mdc(r(x), m(x)) = 1$, portanto $a = r^{-1}(x)$\\
		Passos da divisão:
		\subitem
			$$Dividendo = x^8 + x^4 + x^3 + x + 1$$
			$$Divisor = x^7 + x^4 + x^3 + 1 $$
			$$Quociente = x$$
			$$x \times Divisor = x^8 + x^5 + x^4 + x$$
			$$Resto = x^5 + x^3 + 1$$
		\subitem
			$$Dividendo = x^7 + x^4 + x^3 + 1$$
			$$Divisor = x^5 + x^3 + 1$$
			$$Quociente = x^2 + 1$$
			$$(x^2 + 1) \times Divisor = x^7 + x^3 + x^2 + 1$$
			$$Resto = x^4 + x^2$$	
		\subitem
			$$Dividendo = x^5 + x^3 + 1$$
			$$Divisor = x^4 + x^2$$
			$$Quociente = x$$
			$$x \times Divisor = x^5 + x^3$$
			$$Resto = 1$$	
		\subitem
			$$Dividendo = x^4 + x^2$$
			$$Divisor = 1$$
			$$Quociente = x^4 + x^2$$
			$$(x^4 + x^2) \times Divisor = x^4 + x^2 $$
			$$Resto = 0$$
		Conferimos que $mdc(r(x), m(x)) = 1$, agora acharemos $r^{-1}(x)$\\
		Como queremos $r(x) \times r^{-1}(x) = 1 \ mod \ m(x)$ vamos preencher a tabela do algoritmo de
		euclides estendido até o resto 1.
		\begin{table}[h]
		\centering
		\vspace{0.5cm}
		\begin{tabular}{c|c|c}
			Resto & Quociente & a \\
		\hline                               
			$x^8 + x^4 + x^3 + x + 1$ & $*$                & $0$ \\
			$x^7 + x^4 + x^3 + 1$      & $*$                & $1$ \\
			$x^5 + x^3 + 1$               & $x$                & $x$ \\	
			$x^4 + x^2$                    & $x^2 + 1$       & $x^3 + x + 1$ \\
			$1$                                 & $x$               & $x^4 + x^2$\\
		\end{tabular}
		\end{table}
		\newpage						
		A partir da tabela acima, verificamos que $r^{-1}(x) = x^4 + x^2$, portanto: 
		$$r^{-1}(x)  \ mod \ m(x) = x^4 + x^2$$
	\item
		Verificaremos que: $r^{-1}(x) \otimes r(x) = 1 \ mod \ m(x) $
		$$U(x) = r^{-1}(x) \times r(x)$$
		$$U(x) = (x^4 + x^2) \times (x^7 + x^4 + x^3 + 1)$$
		$$U(x) = x^{11} + x^8 + x^7 + x^4 + x^9 + x^6 + x^5 + x^2$$
		Reescrevendo $U(x)$ temos:
		$$U(x) = x^{11} + x^9 + x^8 + x^7 + x^6 + x^5 + x^4 + x^2$$
		Agora dividiremos $U(x)$ por $m(x)$ para calcular o quociente $Q(x)$ e o resto $R(x)$
		\subitem	
			$$Dividendo = x^{11} + x^9 + x^8 + x^7 + x^6 + x^5 + x^4 + x^2$$
			$$Divisor = x^8 + x^4 + x^3 + x + 1 $$
			$$Quociente = x^3$$
			$$x^3 \times Divisor = x^{11} + x^7 + x^6 + x^4 + x^3$$
			$$Resto = x^9 + x^8 + x^5 + x^3 + x^2$$	
		\subitem	
			$$Dividendo = x^9 + x^8 + x^5 + x^3 + x^2$$
			$$Divisor = x^8 + x^4 + x^3 + x + 1 $$
			$$Quociente = x^3 + x$$
			$$x \times Divisor = x^9 + x^5 + x^4 + x^2 + x$$
			$$Resto = x^8 + x^4 + x^3 + x$$	
		\subitem	
			$$Dividendo = x^8 + x^4 + x^3 + x$$
			$$Divisor = x^8 + x^4 + x^3 + x + 1 $$
			$$Quociente = x^3 + x + 1$$
			$$1 \times Divisor = x^8 + x^4 + x^3 + x + 1$$
			$$Resto = 1$$
		Portanto $Q(x) = x^3 + x + 1$ e $R(x) = 1$	
	\end{enumerate}
	\subsection*{Exercício 3.}
		\begin{enumerate}
			\item
			$A(x)$ e $B(x)$ representado da seguinte forma:\\
			Para $V(x) = [a_3, a_2, a_1, a_0]$
			
			\begin{table}[h]
				\centering
				\vspace{0.5cm}
				\begin{tabular}{c}
				$V(x)$\\
				\hline                               
					$a_3$\\
					$a_2$\\
					$a_1$\\
					$a_0$\\
				\end{tabular}
			\end{table}
			\subitem
				\begin{table}[h]
					\centering
					\vspace{0.5cm}
					\begin{tabular}{c|c}
					$A(x)$ & B(x) \\
					\hline                               
						$10110010$ & $00010010$\\
						$01010101$   & $01111011$ \\
						$10000111$ & $11000100$  \\	
						$00111101$  & $01100110$   \\
					\end{tabular}
				\end{table}
			\newline
			Na forma polinomial temos:
			$$A(x) = \big((x^7 + x^5 + x^4 + x)x^3\big) + \big((x^6 + x^4 + x^2 + 1)x^2\big) + \big((x^7 + x^2 + x + 1)x\big) + \big(x^5 + x^4 + x^3 + x^2 + 1\big)$$
			$$B(x) = \big((x^4 + x)x^3\big) + \big((x^6 + x^5 + x^4 + x^3 + x + 1)x^2\big) + \big((x^7 + x^6 + x^2)x\big) + \big(x^6 + x^5 + x^2 + x\big)$$
			\item
				Agora temos: $C(x) = A(x) \times B(x)$ (abaixo já está calculado o mod $m(x)$ dos termos entre parênteses):\\
				$C(x) = $\\
				$(x^7 + x^5 + x^3 + x)x^6 +$\\
				$(x^5 + x^4 + x^2 + x + 1)x^5 +$\\
				$(x^7 + x^3 + x^2 + 1)x^5 +$\\
				$(x^6 + x^5 + x^4 + x^2 + 1)x^4 +$\\
				$(x^7 + x^4 + x^2)x^4 +$\\
				$(x^7 + x^5 + x^4 + x^3 + x^2 + 1)x^4 +$\\
				$(x^6 + x^3 + 1)x^3 +$\\
				$(x^7 + x^6 + x^5 + x^4)x^3 +$\\
				$(x^7 + x^5 + x^4 + x + 1)x^3 +$\\
				$(x^7 + x^2 + x + 1)x^3 +$\\
				$(x^5 + x^4 + x^2 + x + 1)x^2 +$\\
				$(x^7 + x^3 + x + 1)x^2 +$\\
				$(x^6 + x^5 + x^4 + 1)x^2 +$\\
				$(x^7 + x^6 + x^5 + x)x +$\\
				$(x^7 + x^2)x +$\\
				$(x^7 + x^5 + x^4 + x^2 + x) $\\
				Mostrando $C(x)$ com coeficientes em hexadecimal temos:\\
				$C(x) = (AA)x^6 + (37)x^5 + (8D)x^5 + (75)x^4 + (94)x^4 + (BD)x^4 + (49)x^3 + (F0)x^3 + (B3)x^3 + (87)x^3 + (37)x^2
				+ (8B)x^2 + (71)x^2 + (E2)x + (84)x + (B6)$
			\item
				Agora calcularemos o polinômio resto $R(x)$ resultante da divisão $C(x)/M(x)$ utilizando
				o resultado visto no livro (página 99).
				$$(a_0b_3 + a_1b_2 + a_2b_1 + a_3b_0)x^3 +$$
				$$(a_0b_2 + a_1b_1 + a_2b_0 + a_3b_3)x^2 +$$
				$$(a_0b_1 + a_1b_0 + a_2b_3 + a_3b_2)x +$$
				$$a_0b_0 + a_1b_3 + a_2b_2 + a_3b_1$$
				
				Ficamos com:
				$$(x^7 + x^3 + x^2 + 1)x^3$$
				$$(x^6 + x^5 + x^2 + x + 1)x^2$$
				$$(x^7 + x^6 + x^4 + x^3 + x^2)x$$
				$$(x^7 + x^6 + x^5 + x^3 + x)$$
				Portanto:
				\begin{table}[h]
					\centering
					\vspace{0.5cm}
					\begin{tabular}{c}
					$C(x) mod M(x)$\\
					\hline                               
						$(8D)_{16}$\\
						$(67)_{16}$\\
						$(DC)_{16}$\\
						$(EA)_{16}$\\
					\end{tabular}
				\end{table}
		\end{enumerate}
	\newpage		
	\subsection*{Exercício 4.}
		\begin{enumerate}
			\item	
				Temos $$Ax = (x^7 + x^5 + x^4 + x)x^3 + (x^6 + x^4 + x^2 + 1)x^2 + (x^7 + x^2 + x + 1)x + (x^5 + x^4 + x^3 + x^2 + 1)$$
			\item	
				Agora $C(x) = A(x) \times c(x)$:\\
				$C(x) =$\\ 
				$(x^7 + x^6 + x^3 + x^2 + 1)x^6 +$\\
				$(x^7 + x^5 + x^4 + x)x^5 +$\\
				$(x^7 + x^6 + x^5 + x^4 + x^3 + x^2 + x + 1)x^5 +$\\
				$(x^7 + x^5 + x^4 + x)x^4 +$\\
				$(x^6 + x^4 + x^2 + 1)x^4  +$\\
				$(x^7 + x^4 + x)x^4 +$\\
				$(x^6 + x^5 + x^4 + x^3 + x^2 + x + 1)x^3 +$\\
				$(x^6 + x^4 + x^2 + 1)x^3 +$\\
				$(x^7 + x^2 + x + 1)x^3 +$\\
				$(x^6 + x^2 + x + 1)x^3 +$\\
				$(x^7 + x^5 + x^3 + x)x^2 +$\\
				$(x^7 + x^2 + x + 1)x^2 +$\\
				$(x^5 + x^4 + x^3 + x^2 + 1)x^2 +$\\
				$(x^4 + x^2 + 1)x +$\\
				$(x^5 + x^4 + x^3 + x^2 + 1)x +$\\
				$(x^6 + x^5 + x^4 + x^3 + x)$\\
				
				Em hexadecimal temos:\\
				$C(x) = (CD)x^6 + (B2)x^5 + (FF)x^5 + (B2)x^4 + (55)x^4 + (92)x^4 + (7F)x^3 + (55)x^3 + (87)x^3 + (47)x^3 +
				(AA)x^2 + (87)x^2 + (3D)x^2 + (15)x + (3D)x + (7A)$\\
				\item
					Agora o cálculo de $B(x) = C(x) \% M(x)$:\\
					Usando a fórmula 2.c) da página 99 do livro chegamos em:\\
					\begin{table}[h]
						\centering
						\vspace{0.5cm}
						\begin{tabular}{c}
						$B(x)$\\
						\hline                               
							$(EA)_{16}$\\
							$(DD)_{16}$\\
							$(65)_{16}$\\
							$(0F)_{16}$\\
						\end{tabular}
					\end{table}
				\newpage
				Agora vamos calcular a inversa de $MixColumns(B)$ reproduzindo as operações anteriores com $A(x) = EADD650F$ e
				substituindo $c(x)$ por $c^{-1}(x)$ esperando encontrar como resposta $B(x) = B255873D$\\
				\item
					$$Ax = (x^7 + x^6 + x^5 + x^3 + x^1)x^3 + (x^7 + x^6 + x^4 + x^3 + x^2 + 1)x^2 + (x^6 + x^5 + x^2 + 1)x + (x^3 + 
					x^2 + x + 1)$$
				\item
					$C(x) = $\\
					$(x^5 + x^4 + x^2)x^6 +$\\
					$(x^6 + x^5 + x^4 + x^3 + x^2 + x)x^5 +$\\
					$(x^7 + x^6 + x^3 + x^2 + x)x^5 +$\\
					$(x^7 + x^6 + x^5 + x^4 + x^3 + x + 1)x^4 +$\\
					$(x^5 + x^4 + x^2 + x)x^4 +$\\
					$(x^7 + x^5 + x^3 + x)x^4 +$\\
					$(x^6 + x^4 + x^3 + x + 1)x^3 +$\\
					$(x^6 + x^5 + x^3 + x^2 + x + 1)x^3 +$\\
					$(x^7 + x^6 + x^5 + x^3 + x^2 + x + 1)x^3 +$\\
					$(x^6 + x^5 + x^3 + 1)x^3 +$\\
					$(x^6 + x^3 + x)x^2 +$\\
					$(x^6 + x^5)x^2 +$\\
					$(x^6 + x^3 + x + 1)x^2 +$\\
					$(x^6)x +$\\
					$(x^6 + x^5 + x^4 + x^2 + x + 1)x +$\\
					$(x^6 + x^4 + x^3 + x)$\\	
					
				Em Hexadecimal temos:
				$C(x) = (34)x^6 + (7E)x^5 + (CE)x^5 + (FB)x^4 + (36)x^4 + (AA)x^4 + (5B)x^3 + (6F)x^3 + (EF)x^3 + (69)x^3 +
				(4A)x^2 + (60)x^2 + (4B)x^2 + (40)x + (77)x + (5A)$
				
				\item
					Agora o cálculo de $B(x) = C(x) \% M(x)$:\\
					Usando a fórmula 2.c) da página 99 do livro chegamos em:\\
					\begin{table}[h]
						\centering
						\vspace{0.5cm}
						\begin{tabular}{c}
						$B(x)$\\
						\hline                               
							$(B2)_{16}$\\
							$(55)_{16}$\\
							$(87)_{16}$\\
							$(3D)_{16}$\\
						\end{tabular}
					\end{table}
					
				$\hfill\blacksquare$	
					
		\end{enumerate}			
\end{document}


